% \usepackage{times}
% \usepackage[utf8]{inputenc}
% \usepackage[T1]{fontenc}
\usepackage{amsmath}
\usepackage{amssymb}
\usepackage{hyperref}
\usepackage{url}
\usepackage{booktabs}
\usepackage{amsfonts}
\usepackage{nicefrac}
\usepackage{microtype}
\usepackage{mathtools,bm}
\usepackage{xcolor}
\usepackage{hyperref}
\usepackage{subcaption,placeins}
\usepackage{adjustbox,multirow,dcolumn}
\usepackage{algorithm,algpseudocode,setspace}
\usepackage{xstring}
\usepackage{cleveref}
\usepackage{graphicx}
% \usepackage{natbib}
% \usepackage{natbib}
% \usepackage[numbers]{natbib}
% \usepackage[numbers,sort]{natbib}
% \setcitestyle{sort,comma,numbers,square}
\usepackage{tikz}
\usepackage{pgfplots}
\usepackage{comment}
% \usepackage{xparse}
\usepackage{framed}
\usepackage{pifont}

\makeatletter

% authors
\newcommand{\authornote}[1]{\textsuperscript{\rm {#1}}}
\newcommand{\email}[1]{\href{mailto:#1}{\texttt{#1}}}
\newcommand*\samethanks[1][\value{footnote}]{%
    \footnotemark[#1]\hspace{0.4em}}

% notes
\newcommand{\fixme}{\marginpar{FIXME}}
\newcommand{\todo}{\marginpar{TODO}}
% \newcommand{\rebuttal}[1]{{\color{blue}{#1}}}
\newcommand{\revise}[1]{{\color{blue}{#1}}}

% math
\input{math_commands}

% units
\newcommand{\mega}{\,\textrm{M}}
\newcommand{\giga}{\,\textrm{G}}
\newcommand{\tera}{\,\textrm{T}}
\newcommand{\peta}{\,\textrm{P}}

% latin
\DeclareRobustCommand\onedot{\futurelet\@let@token\@onedot}
\def\@onedot{\ifx\@let@token.\else.\null\fi}
\newcommand{\Eg}{\emph{E.g\@\onedot}}
\newcommand{\eg}{\emph{e.g\@\onedot}}
\newcommand{\etc}{\emph{etc\@\onedot}}
\newcommand{\etal}{\emph{et~al\@\onedot}}
\newcommand{\Ie}{\emph{I.e\@\onedot}}
\newcommand{\ie}{\emph{i.e\@\onedot}}
\newcommand{\versus}{\texorpdfstring{\emph{vs\@\onedot}}{vs.}}
\newcommand{\viceversa}{\emph{vice versa}}
\newcommand{\wrt}{\emph{w.r.t\@\onedot}}
\newcommand{\subto}{\emph{s.t\@\onedot}}
\newcommand{\numero}[1]{No\@.\ {#1}}
\newcommand{\numeros}[1]{Nos\@.\ {#1}}
\newcommand{\first}{1\textsuperscript{st}}
\newcommand{\second}{2\textsuperscript{nd}}
\newcommand{\third}{3\textsuperscript{rd}}
\newcommand{\ordinal}[1]{{#1}\textsuperscript{th}}

% figures
\graphicspath{{figures/}}
\DeclareRobustCommand{\inlinegraphics}[2][1]{%
  \begingroup\normalfont%
  \includegraphics[height=#1\fontcharht\font`\B]{#2}%
  \endgroup
}

% tikz
\usetikzlibrary{external}
\usepgfplotslibrary{colorbrewer}
\pgfplotsset{compat=newest}
\tikzexternalize[prefix=tikz/]
\newcommand{\tikzsnfn}[1]{%
    \def\fn{#1}\StrSubstitute{\fn}{/}{_}[\fn]%
    \tikzsetnextfilename{\fn}}
% \newtoggle{final}\togglefalse{final}
% \tikzifexternalizing{\toggletrue{final}}{\togglefalse{final}}

% tables
\newcolumntype{d}{D{.}{.}{3.2}}
\newcolumntype{B}{>{\boldmath\DC@{.}{.}{3.2}}c<{\DC@end}}
\newcommand{\supstar}{\mbox{\textsuperscript{\( \star \)}}}
\newcommand{\supcirc}{\mbox{\textsuperscript{\( \circ \)}}}
\newcommand{\supdagger}{\textsuperscript{\textdagger}}
\newcommand{\supddagger}{\textsuperscript{\textdaggerdbl}}
\newcommand{\tbox}[1]{\begin{tabular}[c]{@{}c@{}}#1\end{tabular}}
\newcommand{\tcenter}[1]{\multicolumn{1}{c}{{#1}}}
\newcommand{\thead}[1]{\multicolumn{1}{c}{\textbf{#1}}}
\newcommand{\tbnum}[1]{\multicolumn{1}{B}{#1}}
\newcommand{\tna}{\tcenter{---}}
\newcommand{\tpm}[2]{\( {#1}{\scriptstyle \pm #2} \)}
\newcommand{\tbpm}[2]{\tpm{\mathbf{#1}}{#2}}
\newcommand{\tbgreen}[1]{\textcolor[rgb]{0,0.502,0}{#1}}
\newcommand{\tbred}[1]{\textcolor[rgb]{0.502,0,0}{#1}}

% macros
\newcommand{\numfirstorder}{6}
\newcommand{\numsecondorder}{2}
\newcommand{\nummetrics}{8}
% TODO: a better name
\newcommand{\MethodVerb}{{mixSGA}}
\newcommand{\Method}{\emph{\MethodVerb}}
\newcommand{\idx}[2]{{#1}^{\scriptscriptstyle[{#2}]}}

% math
\DeclarePairedDelimiter{\parens}{\lparen}{\rparen}
\DeclarePairedDelimiter{\angles}{\langle}{\rangle}
\DeclarePairedDelimiter{\bracks}{[}{]}
\DeclarePairedDelimiter{\braces}{\{}{\}}
\DeclarePairedDelimiter{\verts}{\lvert}{\rvert}
\DeclarePairedDelimiter{\Verts}{\lVert}{\rVert}
\DeclarePairedDelimiter{\floors}{\lfloor}{\rfloor}
\DeclarePairedDelimiter{\ceils}{\lceil}{\rceil}
\newcommand{\expect}[2][]{\mathbb{E}_{#1}\bracks*{#2}}
\newcommand{\diff}[2]{\frac{\partial{#1}}{\partial{#2}}}
\newcommand{\shat}[1]{\vphantom{#1}\smash[t]{\hat{#1}}}
\newcommand{\x}{{\mathbf{x}}}
\newcommand{\z}{{\mathbf{z}}}
\newcommand{\y}{\mathbf{y}}
\newcommand{\ratios}{\bm{\rho}}
\newcommand{\ratio}{\rho}
\newcommand{\mask}{\mathbf{m}}
\newcommand{\weight}{{\bm\theta}}
\newcommand{\keyweight}{{\mathbf{w}^\textrm{k}}}
\newcommand{\valueweight}{{\mathbf{w}^\textrm{v}}}
\newcommand{\keybias}{{\mathbf{b}^\textrm{k}}}
\newcommand{\valuebias}{{\mathbf{b}^\textrm{v}}}
\newcommand{\routeweight}{\bm\phi}
\newcommand{\routebias}{\bm\beta}
\newcommand{\dataset}{\mathcal{D}}
\newcommand{\testset}{\dataset_\textrm{test}}
\newcommand{\trainset}{\dataset_\textrm{train}}
\newcommand{\realset}{\mathbb{R}}
\newcommand{\inputset}{\mathcal{I}}
\newcommand{\classset}{\mathcal{C}}
\newcommand{\outputset}{\realset^K}
\newcommand{\stepsize}{\alpha}
\newcommand{\momentum}{\nu}
\DeclareMathOperator{\topk}{\mathsf{top}}
\DeclareMathOperator{\bigO}{\mathcal{O}}
\DeclareMathOperator{\loss}{\mathcal{L}}
\DeclareMathOperator{\sceloss}{\mathcal{L}^\textrm{sce}}
\DeclareMathOperator{\auxloss}{\mathcal{L}_\textrm{aux}}
\DeclareMathOperator{\modelloss}{\mathcal{L}_\textrm{model}}
\DeclareMathOperator{\dist}{{dist}}
\DeclareMathOperator{\norm}{{norm}}
\DeclareMathOperator{\moe}{\mathsf{moe}}
\DeclareMathOperator{\onehot}{{onehot}}
\DeclareMathOperator{\indicator}{\mathbf{1}}
\DeclareMathOperator{\betadist}{\mathcal{B}}
\DeclareMathOperator{\sample}{sample}
\DeclareMathOperator{\batchsample}{minibatch}
% \DeclareMathOperator{\sigmoid}{{\sigma}}
% \DeclareMathOperator{\softmax}{\mathsf{softmax}}
\DeclareMathOperator{\routescore}{\mathsf{S}}
\DeclareMathOperator{\recon}{\mathsf{Reconstruct}}
\DeclareMathOperator{\bceloss}{\mathsf{BCE}}
% plots
% \newcommand{\hpad}{\hspace*{\fill}}
% \newcommand{\subplotsizeslack}{1}
% \NewDocumentCommand{\subplot}{ O{1} O{} m m m }{%
%     \hpad%
%     \begin{subfigure}{(\textwidth / #1) * \subplotsizeslack}
%         \includegraphics[width=\textwidth, #2]{#3}
%         \caption{#4}\label{#5}
%     \end{subfigure}%
% }

\makeatother
