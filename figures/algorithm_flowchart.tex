\begin{figure*}[t]
\centering
\begin{tikzpicture}[
    node distance = 2cm,
    startstop/.style = {rectangle, rounded corners, minimum width=3cm, minimum height=1cm, text centered, draw=black, fill=red!30},
    process/.style = {rectangle, minimum width=3cm, minimum height=1cm, text centered, draw=black, fill=orange!30},
    decision/.style = {diamond, minimum width=3cm, minimum height=1cm, text centered, draw=black, fill=green!30},
    arrow/.style = {thick,->,>=stealth}
]

% 起始节点
\node (start) [startstop] {Start: Task Context $u$};

% 初始化
\node (init) [process, below of=start] {Initialize Population\\$\mathcal{P} \leftarrow \{\text{InitGen}(u)\}_{i=1}^{m}$};

% 迭代开始
\node (loop) [process, below of=init] {Iteration $t = 1$ to $n$};

% 评估适应度
\node (evaluate) [process, below of=loop] {Compute Fitness\\$F(T)$ for each $T \in \mathcal{P}$};

% 选择top-k
\node (select) [process, below of=evaluate] {Select Top-k\\$\mathcal{S} \leftarrow \text{top-}k(\mathcal{P}, F)$};

% 找到最优
\node (best) [process, below of=select] {Find Best Individual\\$T^* \leftarrow \arg\max_{T\in\mathcal{S}} F(T)$};

% 计算嵌入
\node (embed) [process, below of=best] {Compute Embeddings\\$e_T \leftarrow \mathrm{Embed}(T)$ for $T\in\mathcal{S}$};

% 语义最远距离选择
\node (farthest) [process, below of=embed] {Select Farthest Individual\\$T_j \leftarrow \arg\max_{T\in\mathcal{S}\setminus\{T^*\}} \mathrm{dist}(e_T, e_{T^*})$};

% 交叉操作
\node (crossover) [process, below of=farthest] {Semantic Crossover\\offspring $\leftarrow \mathcal{C}(T^*, T_j)$\\[LLM-based validity check]};

% 更新种群
\node (update) [process, below of=crossover] {Update Population\\$\mathcal{P} \leftarrow \mathcal{P} \cup \{\text{offspring}\}$};

% 决策点
\node (decision) [decision, right of=loop, xshift=4cm] {$t < n$?};

% 最终结果
\node (final) [process, right of=best, xshift=4cm] {Select Best\\$T_{\text{adv}}^* \leftarrow \arg\max_{T\in\mathcal{P}} F(T)$};

% 结束
\node (end) [startstop, right of=final, xshift=3cm] {Return $T_{\text{adv}}^*$};

% 箭头连接
\draw [arrow] (start) -- (init);
\draw [arrow] (init) -- (loop);
\draw [arrow] (loop) -- (evaluate);
\draw [arrow] (evaluate) -- (select);
\draw [arrow] (select) -- (best);
\draw [arrow] (best) -- (embed);
\draw [arrow] (embed) -- (farthest);
\draw [arrow] (farthest) -- (crossover);
\draw [arrow] (crossover) -- (update);
\draw [arrow] (update) -- (decision);
\draw [arrow] (decision) -- node[anchor=south] {Yes} (loop);
\draw [arrow] (decision) -- node[anchor=west] {No} (final);
\draw [arrow] (final) -- (end);

% 标注关键步骤
\node [above right] at (crossover.north east) {\small \textbf{Key Step}};
\node [below right, text width=3cm] at (farthest.south east) {\small \emph{Semantic diversity exploration}};

\end{tikzpicture}
\caption{\methodname{} Algorithm Flowchart. The algorithm iteratively evolves adversarial tools by balancing exploitation of high-fitness individuals (T*) with exploration of semantically diverse candidates (Tj).}}
\label{fig:algorithm-flowchart}
\end{figure*}