\begin{figure*}[t]
\centering
\begin{tikzpicture}[
    node distance = 2.2cm,
    auto,
    font=\small,
    startstop/.style = {rectangle, rounded corners=8pt, minimum width=3.5cm, minimum height=1cm, text centered, draw=black, fill=blue!30, thick},
    process/.style = {rectangle, minimum width=3.5cm, minimum height=1cm, text centered, draw=black, fill=yellow!30, thick},
    decision/.style = {diamond, minimum width=3cm, minimum height=1cm, text centered, draw=black, fill=green!30, thick},
    io/.style = {trapezium, trapezium left angle=70, trapezium right angle=110, minimum width=3cm, minimum height=1cm, text centered, draw=black, fill=cyan!30, thick},
    arrow/.style = {thick,->,>=stealth, line width=1.2pt},
    emph/.style = {rectangle, minimum width=3.5cm, minimum height=1cm, text centered, draw=red, fill=magenta!20, thick, dashed}
]

% 颜色定义
\definecolor{lightblue}{RGB}{173,216,230}
\definecolor{lightgreen}{RGB}{144,238,144}
\definecolor{lightsalmon}{RGB}{255,160,122}
\definecolor{lightyellow}{RGB}{255,255,224}

% 起始输入
\node (input) [io] {Input: Task Context $u$, Seed size $m$, Iterations $n$};

% 初始化
\node (init) [process, below of=input, yshift=-0.5cm] {
    \textbf{Initialize Population}\\
    $\mathcal{P} \leftarrow \{\text{InitGen}(u)\}_{i=1}^{m}$
};

% 迭代框背景
\node (iterationbox) [draw=gray!50, dashed, rounded corners=10pt, fit=(current bounding box.south west)(current bounding box.south east), inner sep=10pt, minimum height=15cm] {};
\node at (iterationbox.north) [above=5pt] {\large \textbf{\textcolor{gray!70}{Evolution Loop (1 to n iterations}}};

% 评估适应度
\node (evaluate) [process, below of=init, yshift=-0.5cm] {
    \textbf{Fitness Evaluation}\\
    Compute $F(T)$ for each $T \in \mathcal{P}$\\
    [Objective: Resource exhaustion / Task failure]
};

% 选择操作
\node (select) [process, below of=evaluate] {
    \textbf{Selection}\\
    $\mathcal{S} \leftarrow \text{top-}k(\mathcal{P}, F)$
};

% 找到最优个体
\node (best) [process, below of=select] {
    \textbf{Best Individual}\\
    $T^* \leftarrow \arg\max_{T\in\mathcal{S}} F(T)$\\
    [Exploitation]
};

% 语义嵌入
\node (embed) [process, below of=best] {
    \textbf{Semantic Embedding}\\
    $e_T \leftarrow \mathrm{Embed}(T)$ for $T\in\mathcal{S}$
};

% 语义最远距离(关键创新点)
\node (farthest) [emph, below of=embed] {
    \textbf{Farthest Individual Selection}\\
    $T_j \leftarrow \arg\max_{T\in\mathcal{S}\setminus\{T^*\}} \mathrm{dist}(e_T, e_{T^*})$\\
    [\textcolor{red}{\textbf{Key Innovation: Semantic Diversity}}]
};

% 语义交叉(核心步骤)
\node (crossover) [emph, below of=farthest] {
    \textbf{Semantic Crossover}\\
    offspring $\leftarrow \mathcal{C}(T^*, T_j)$\\
    \textcolor{red}{LLM-based Validity Check}
};

% 更新种群
\node (update) [process, below of=crossover] {
    \textbf{Update Population}\\
    $\mathcal{P} \leftarrow \mathcal{P} \cup \{\text{offspring}\}$\\
    Optional: Prune / Re-rank
};

% 决策节点
\node (decision) [decision, right of=evaluate, xshift=5.5cm] {
    Iteration\\
    $t < n$?
};

% 最终选择
\node (final) [io, right of=best, xshift=5.5cm] {
    \textbf{Final Selection}\\
    $T_{\text{adv}}^* \leftarrow \arg\max_{T\in\mathcal{P}} F(T)$
};

% 输出
\node (output) [io, below of=final] {
    \textbf{Output}\\
    Best Adversarial Tool $T_{\text{adv}}^*$
};

% 完成
\node (end) [startstop, below of=output] {
    Attack Evaluation on MCP Agent
};

% 背景框 - 关键创新区域
\begin{scope}[on background layer]
\node[circle, draw=red!70, dashed, minimum size=6cm, fill=red!5] at (farthest) {};
\node[above=0.2cm of farthest.north, red] {\small \textbf{Key Innovation Area}};
\end{scope}

% 箭头连接 - 主流程
\draw [arrow] (input) -- (init);
\draw [arrow] (init) -- (evaluate);
\draw [arrow] (evaluate) -- (select);
\draw [arrow] (select) -- (best);
\draw [arrow] (best) -- (embed);
\draw [arrow] (embed) -- (farthest);
\draw [arrow] (farthest) -- (crossover);
\draw [arrow] (crossover) -- (update);
\draw [arrow] (update) -- +(3,0) |- (decision);
\draw [arrow] (decision) -- node[anchor=south] {\textbf{Yes}} +(0,2) |- (evaluate);
\draw [arrow] (decision) -- node[anchor=west] {\textbf{No}} (final);
\draw [arrow] (final) -- (output);
\draw [arrow] (output) -- (end);

% 辅助说明箭头
\draw [arrow, bend left=30, dashed, gray] (best.east) -- node[anchor=south, sloped] {\tiny exploitation} (farthest.east);
\draw [arrow, bend right=30, dashed, gray] (farthest.west) -- node[anchor=north, sloped] {\tiny exploration} (crossover.west);

% 标注关键概念
\node[above right=0.3cm and 0.1cm of farthest.east, text width=4cm, align=left] {\small \textcolor{red}{\textbf{Semantic Diversity:}}\\
\small Balance local search (T*) with global exploration (Tj)};

\node[below=0.5cm of update.south, text width=6cm, align=center] {\small \textcolor{blue}{Population maintains diversity while improving fitness}};

\end{tikzpicture}
\caption{Algorithm Flowchart: SMTH for Adversarial MCP Tool Generation.
The algorithm balances exploitation of high-fitness individuals with exploration of semantically diverse candidates through a genetic evolution framework.
Key innovation lies in the semantic distance selection ($T_j$) that ensures diversity in the tool population.}}
\label{fig:algorithm-flowchart-advanced}
\end{figure*}